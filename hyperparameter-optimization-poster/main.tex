%%%%%%%%%%%%%%%%%%%%%%%%%%%%%%%%%%%%%%%%%
% MUW Poster
% LaTeX Template
% Version 1.0 (31/08/2016)
% (Based on Version 1.0 (31/08/2015) of the Jacobs Portrait Poster
%
% License:
% CC BY-NC-SA 3.0 (http://creativecommons.org/licenses/by-nc-sa/3.0/)
%
% Created by:
% Nicolas Ballarini, CeMSIIS, Medical University of Vienna
% nicoballarini@gmail.com
% http://statistics.msi.meduniwien.ac.at/
%%%%%%%%%%%%%%%%%%%%%%%%%%%%%%%%%%%%%%%%%

\def\footer#1{\def\insertfooter{#1}}
%--------------------------------------------------------------------------------------
%	PACKAGES AND OTHER DOCUMENT CONFIGURATIONS
%--------------------------------------------------------------------------------------

\documentclass[final]{beamer}

\usepackage[scale=1.150]{beamerposter} % Use the beamerposter package
\usetheme{MUWposter} % Use the MUWposter theme supplied with this template

\usepackage{multicol}
\usepackage{array}
\usepackage{pgf}  
\usepackage{mathtools}
\usepackage{amsmath, amsthm, amssymb, amsfonts}
\usepackage{exscale}
\usepackage{xcolor}
\usepackage{ushort}
\usepackage{setspace}
\usepackage[square,numbers]{natbib}
\usepackage{url}
\bibliographystyle{abbrvnat}
\renewcommand{\vec}[1]{\ushort{#1}}
\renewcommand{\vec}[1]{\mathbf{#1}}
\definecolor{greenMUW}{RGB}{60,191,174}
\definecolor{blueMUW}{RGB}{17,29,79}
\definecolor{skinMUW}{RGB}{254,228,217}
\definecolor{hellblauMUW}{RGB}{95,180,229}

%-----------------------------------------------
%  START Set the colors
%-----------------------------------------------
\colorlet{themecolor}{greenMUW}
\usebackgroundtemplate{\includegraphics{MUW_green.pdf}}  %MARCO

%-----------------------------------------------
%  END Set the colors
%-----------------------------------------------

%-----------------------------------------------
%  START Set the width of the columns
%-----------------------------------------------
\setlength{\paperwidth}{33.1in} % A0 width: 46.8in
\setlength{\paperheight}{46.8in} % A0 height: 33.1in
\newlength{\sepmargin}
\newlength{\sepwid}
\newlength{\onecolwid}
\newlength{\twocolwid}
\newlength{\threecolwid}

% The following measures are used for 2 columns
\setlength{\sepmargin}{0.055\paperwidth} % Separation width (white space) between columns
\setlength{\sepwid}{0.03\paperwidth} % Separation width (white space) between columns
\setlength{\onecolwid}{0.43\paperwidth} % Width of one column
\setlength{\twocolwid}{0.9\paperwidth} % Width of two columns

%-----------------------------------------------
%  END Set the width of the columns
%-----------------------------------------------

%--------------------------------------------------------------------------------------
%	TITLE SECTION 
%----------------------------------------------------------------------------------------
\setbeamertemplate{title}[left]
\setbeamertemplate{frametitle}[default][left]

\title{Optimización de Hiperparámetros} % Poster title

\author{Círculo de Data Science} % Author(s)

\institute{Universidad Nacional de Ingeniería} % Institution(s)
%--------------------------------------------------------------------------------------


\begin{document}

  \addtobeamertemplate{block end}{}{\vspace*{1ex}} % White space under blocks
  \addtobeamertemplate{block alerted end}{}{\vspace*{0ex}} % White space under highlighted (alert) blocks
  \setlength{\belowcaptionskip}{2ex} % White space under figures
  \setlength\belowdisplayshortskip{1ex} % White space under equations
  
  
  \begin{frame}[t] % The whole poster is enclosed in one beamer frame

      \begin{columns}[t] % The whole poster consists of two major columns
	  
      \begin{column}{\sepmargin}\end{column}
      
	    \begin{column}{\onecolwid} % The first column

		  \begin{block}{¿Qué es la optimización de Hiperparámetros?}
          Cuando se entrenan modelos de machine learning, cada conjunto de datos y cada modelo necesitan un conjunto diferente de hiperparámetros, que son un tipo de variables. La única forma de determinarlos es mediante la realización de múltiples experimentos, en los que se elige un conjunto de hiperparámetros y se los ejecuta a través del modelo. Esto se denomina ajuste de hiperparámetros. Básicamente, está entrenando su modelo secuencialmente con diferentes conjuntos de hiperparámetros. \\
          \\
          
          \\ 
          Este proceso puede ser manual o puede elegir uno de los distintos métodos automatizados de ajuste de hiperparámetros que presentaremos a continuación.
          
          \end{block}
                
          \begin{block}{Ajuste manual}
          Este es el proceso de seleccionar y ajustar hiperparámetros basándose en la experiencia y la intuición del desarrollador. Es un método simple que requiere iterar varias veces, ajustando los hiperparámetros y observando el impacto en el rendimiento del modelo. Aunque el ajuste manual puede ser efectivo para ajustes rápidos y modelos simples, es subjetivo y puede ser ineficiente para espacios de hiperparámetros grandes o modelos complejos.


          \end{block}
          
          \begin{block}{Búsqueda en Cuadrícula (Grid Search)}
          Este método implica definir un conjunto de valores posibles para cada hiperparámetro y luego evaluar todas las combinaciones posibles de estos valores. La búsqueda en cuadrícula garantiza que se explore exhaustivamente el espacio de hiperparámetros definido, pero puede ser muy costosa en términos de tiempo y recursos computacionales, especialmente cuando el número de hiperparámetros y sus posibles valores es grande.
          
                \begin{figure}
                    \includegraphics[width=1\linewidth]{busqueda_cuadricula.png}
				\end{figure}          

          \end{block}
          
         \end{column}   %%ACABAMOS LA 1RA COLUMNA
                  
                  
                  
         \begin{column}{\sepwid}  \end{column} %%
         
         
         
         \begin{column}{\onecolwid} %The second column
         
         \begin{block}{Búsqueda Aleatoria (Random Search)}

          A diferencia de la búsqueda en cuadrícula, la búsqueda aleatoria selecciona combinaciones de hiperparámetros de forma aleatoria dentro de los rangos definidos para cada hiperparámetro. Este método puede ser más eficiente que la búsqueda en cuadrícula, especialmente en casos donde el espacio de hiperparámetros es grande, ya que no todas las combinaciones son evaluadas y, por tanto, requiere menos tiempo y recursos para encontrar una combinación satisfactoria de hiperparámetros.

                \begin{figure}
                    \includegraphics[width=1\linewidth]{busqueda_aleatoria.png}
				\end{figure}   
    
          \end{block}
%%%%%%%%%%%%%%%%%%%%%%%%%%%%%%%%%%%%%%%%%%%%%%%%%%%%%%%%%%%%%%%%%%55

\begin{block}{Optimización Basada en Modelos Bayesianos}

          Este enfoque utiliza técnicas probabilísticas para modelar el rendimiento del modelo en función de los hiperparámetros y utiliza esta información para guiar la búsqueda hacia las combinaciones de hiperparámetros más prometedoras. La optimización basada en modelos bayesianos es más sofisticada y puede ser más eficaz que los métodos anteriores, especialmente en espacios de hiperparámetros complejos, ya que se enfoca en explorar áreas del espacio de hiperparámetros que tienen más probabilidades de mejorar el rendimiento del modelo.

                \begin{figure}
                    \includegraphics[width=1\linewidth]{optimizacion_bayesiana.png}
				\end{figure}   
    
          \end{block}

%%%%%%%%%%%%%%%%%%%%%%%%%%%%%%%%%%%%%%%%%%%%%%%%%%%%%%%%%%          


      \end{column}
      
      \begin{column}{\sepmargin} \end{column}
      \end{columns} 
       
      \begin{columns}[t] % Split up the two columns wide column again
      
      \begin{column}{\sepmargin} \end{column}
        \begin{column}{\onecolwid} % The first column

				
		    \end{column} % End of the first column
			\begin{column}{\sepwid}\end{column} % Empty spacer column
			
            
			\begin{column}{\sepmargin}\end{column} % Empty spacer column
            
\end{columns} % End of all the columns in the poster

\end{frame} % End of the enclosing frame
	
\end{document}

